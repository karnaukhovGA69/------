\documentclass[a4paper,12pt]{article}
\usepackage[utf8]{inputenc}
\usepackage[russian]{babel}
\usepackage{amsmath, amssymb}

\title{Асимптотический анализ алгоритмов}
\date{\today}
\author{}

\begin{document}
\maketitle

\section*{Algorithm 1}

Рассмотрим рекуррентное соотношение:
\[
T(n) = T(n - 5) + T(n - 8) + n^2 \cdot c, \text{ где } c - \text{константа.}
\]

Так как у нас два различных рекурсивных подвызыва, то использовать мастер-теорему невозможно.  
Также нельзя применять метод Акры–Баззи, так как аргумент уменьшается не делением, а вычитанием константы.  
Следовательно, необходимо определить верхнюю и нижнюю границы временной сложности.

\subsection*{Верхняя граница}

Так как при каждом рекурсивном вызове количество вызовов увеличивается в два раза,  
то на уровне рекурсии \(i\) будет \(2^i\) подзадач.  
Аргумент уменьшается минимум на 5, поэтому глубина дерева рекурсии равна примерно \(n / 5\).  
Каждый узел вносит вклад \(O(n^2)\), следовательно:
\[
T(n) = O(2^{n/5} \cdot n^2).
\]

\subsection*{Нижняя граница}

Теперь возьмём глубину рекурсии как \(n / 8\), так как аргумент уменьшается максимум на 8.  
Количество вызовов на каждом уровне также удваивается,  
а затраты на каждом уровне составляют примерно \((n / 2)^2\).  
Таким образом, получаем:
\[
T(n) = \Omega(2^{n/8} \cdot n^2).
\]

\section*{Algorithm 2}

Рассмотрим рекуррентное соотношение:
\[
T(n) = 2T(n / 4) + \frac{n}{3}.
\]

Применим мастер-теорему:
\[
a = 2 \ge 1, \quad b = 4 > 1, \quad k = 1 > 0, \quad f(n) \text{ — монотонна.}
\]
\[
\log_b a = \log_4 2 = \frac{1}{2} < k = 1.
\]

Следовательно, согласно первому случаю мастер-теоремы:
\[
T(n) = O(n).
\]

\end{document}
